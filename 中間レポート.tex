\documentclass[a4paper, 11pt, titlepage]{jsarticle}
\usepackage[dvipdfmx]{graphicx}
\usepackage{listings}
\usepackage{amsmath}
\usepackage{url}

\title{知能情報実験III(データマイニング班)\\Twitter上のテキスト文を対象とした「コロナで何が困っているのか」を見つける}
\author{グループの学籍番号 205759A, 205720E, 205763J, 195719J}
\date{提出日:2022年6月9日}
\begin{document}
\maketitle
\tableofcontents
\clearpage

\abstract{概要}
本文書は知能情報実験III(データマイニング班)におけるレポートのテンプレートとして用意したものである。
一般的な実験レポートに関する補足と共に、データマイニング班における実験レポートに求められる内容を確認するために用意した。
ここに書いてある事柄は全てを必須とするわけではなく、適宜取捨選択や追加編集してもらって構わないが、実験報告書としての位置づけを忘れずに利用すること。

\section{はじめに}

\subsection{実験の目的と達成目標}
知能情報実験IIIは、情報工学分野のより専門的な知識を理解・習得することを目的として、半年間でシステムの開発やデータ解析等に取り組む実施される。
その中の一つデータマイニング班においては機械学習外観ならびにその応用を通し、対象問題への理解、特徴量抽出等の前処理、バージョン管理やデバッグ・テスト等を含む仕様が定まっていない状況下における開発方法、コード解説や実験再現のためのドキュメント作成等の習得を目指す。

\subsection{テーマ**とは}
本グループではTwitter上のテキスト文を対象とした「コロナで何が困っているのか」を見つけることを対象問題として設定した.コロナで何が困っているのかがわかることで,その後の改善策を見出すことができ、今後に応用できるのではないか.

\section{実験方法}
実験手順を過去形で述べよう。日誌のように時系列ではなく、成果物として報告する最終版を再現するための実験手順で良い。
第三者が再現するために必要な手順であることが重要だ。
また、列挙した項目毎に具体的な内容をsubsectionで述べよう。


\begin{enumerate}
 \item 実験目的
 あ
 \item 実験計画
 \item データセット構築
 \item モデルの選定
 \item パラメータ調整
\end{enumerate}

\subsection{実験目的}
実験を通して明らかにしたいこと、確認したいこと、検証したいことを述べよう。

\subsection{データセット構築}
既にどこかで公開されているデータセットをダウンロードして利用したのならば、そのURLを掲載する程度で構いません。
独自構築した場合にはその構築方法を述べよう。

\subsection{モデル選定}
どのようなモデルやアルゴリズムを利用したのか、何故それを選んだのか述べよう。

\subsection{パラメータ調整}
手動調整が必要なパラメータについて、どのように調整したのか述べよう。


\section{実験結果}
事実として得られた結果を示そう。
なお、以下の点に留意すること。

\begin{itemize}
 \item 「思う」「思われる」のような主観ではなく、客観的事実を述べること。
 \item 図表には適切なキャプションを付けること。
 \item 挿入した図表について、本文中でその読み方を述べること。その際にはlabel, refにより相互参照すること。
 \item レポートにおけるグラフの作成においては、以下の点に注意する。
 \begin{itemize}
 	\item 軸目盛および軸ラベルに関する注意事項
 	\begin{itemize}
 		\item 必ず軸ラベルを表示する
 		\item 軸に単位がある場合には、ラベルに単位を付記する
 		\item 軸目盛は適切な感覚で表示する
 		\item 軸目盛は述べたい内容に応じて線形スケールとlogスケールを使い分ける
 		\item 印刷時に明瞭に読むことができるサイズで表示する
 	\end{itemize}
 	\item 線・点・ポイントおよび凡例に関する注意事項
 	\begin{itemize}
 		\item 線・点・ポイントは、印刷時に明瞭に識別できる太さやサイズで表示する
 		\item 1つのグラフに複数にデータを表示する際には、データごとに異なる線種、線の太さ、ポイント形状などを使用する
 		\item モノクロ印刷でも識別できるように線・点・ポイントを使用することが望ましい
 		\item 凡例は線・点・ポイントに重ならないように注意する
 	\end{itemize}
 \end{itemize}
\end{itemize}

\section{考察}
実験課題への取り組みを通し、実験の意義、実験からわかったこと、今後の展望などを述べる。
失敗やつまづきがあれば、それらについての失敗分析を含めると良い。

\section{意図していた実験計画との違い}
グループワークとして2ヶ月程度の時間が用意されていた。
ガントチャート\ref{ganttchart}等、何かしら工夫して全体の計画を述べよう。
これらの期間をどのように使おうとし、実際どうだったのかについて自己評価(振り返り)してみよう。
大きなズレがある場合それは何故起きたのか、どうやればそのギャップを縮められそうか検討してみよう。

\section{まとめ}
データマイニング班の達成目標を振り返り、選んだテーマに対する機械学習の適用を通して得られた知見や学んだことをまとめよう。
また今後やるべきことや後進に伝えたいこと等あれば自由に述べよう。

\begin{thebibliography}{n}
  \bibitem{kanazawa}レポート作成の手引き レポートの基本的形式に関するガイド, \url{https://www.kanazawa-u.ac.jp/wp-content/uploads/2015/01/tebiki2.pdf}, 2020/07/02.
	\bibitem{theme1}テーマ出典, 書籍or特定の記事or webpage, webpageの場合は参照日も記そう, 2020/07/02.
	\bibitem{theme2}テーマ出典2, 出典は半角,.で書こう.
	\bibitem{ganttchart}ガントチャート, \url{https://ja.wikipedia.org/wiki/ガントチャート}, 2020/07/02.
\end{thebibliography}
\end{document}
